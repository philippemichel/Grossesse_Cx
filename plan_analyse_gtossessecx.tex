% Options for packages loaded elsewhere
\PassOptionsToPackage{unicode}{hyperref}
\PassOptionsToPackage{hyphens}{url}
\PassOptionsToPackage{dvipsnames,svgnames,x11names}{xcolor}
%
\documentclass[
  a4paperpaper,
  french]{scrartcl}

\usepackage{amsmath,amssymb}
\usepackage{lmodern}
\usepackage{iftex}
\ifPDFTeX
  \usepackage[T1]{fontenc}
  \usepackage[utf8]{inputenc}
  \usepackage{textcomp} % provide euro and other symbols
\else % if luatex or xetex
  \usepackage{unicode-math}
  \defaultfontfeatures{Scale=MatchLowercase}
  \defaultfontfeatures[\rmfamily]{Ligatures=TeX,Scale=1}
  \setmainfont[Numbers=OldStyle,Ligatures=TeX,Ligatures= Rare,Ligatures=
Historic,Style=Alternate,Contextuals = Swash]{Minion Pro}
  \setsansfont[Ligatures=TeX]{Myriad Pro}
\fi
% Use upquote if available, for straight quotes in verbatim environments
\IfFileExists{upquote.sty}{\usepackage{upquote}}{}
\IfFileExists{microtype.sty}{% use microtype if available
  \usepackage[]{microtype}
  \UseMicrotypeSet[protrusion]{basicmath} % disable protrusion for tt fonts
}{}
\makeatletter
\@ifundefined{KOMAClassName}{% if non-KOMA class
  \IfFileExists{parskip.sty}{%
    \usepackage{parskip}
  }{% else
    \setlength{\parindent}{0pt}
    \setlength{\parskip}{6pt plus 2pt minus 1pt}}
}{% if KOMA class
  \KOMAoptions{parskip=half}}
\makeatother
\usepackage{xcolor}
\setlength{\emergencystretch}{3em} % prevent overfull lines
\setcounter{secnumdepth}{5}
% Make \paragraph and \subparagraph free-standing
\ifx\paragraph\undefined\else
  \let\oldparagraph\paragraph
  \renewcommand{\paragraph}[1]{\oldparagraph{#1}\mbox{}}
\fi
\ifx\subparagraph\undefined\else
  \let\oldsubparagraph\subparagraph
  \renewcommand{\subparagraph}[1]{\oldsubparagraph{#1}\mbox{}}
\fi


\providecommand{\tightlist}{%
  \setlength{\itemsep}{0pt}\setlength{\parskip}{0pt}}\usepackage{longtable,booktabs,array}
\usepackage{calc} % for calculating minipage widths
% Correct order of tables after \paragraph or \subparagraph
\usepackage{etoolbox}
\makeatletter
\patchcmd\longtable{\par}{\if@noskipsec\mbox{}\fi\par}{}{}
\makeatother
% Allow footnotes in longtable head/foot
\IfFileExists{footnotehyper.sty}{\usepackage{footnotehyper}}{\usepackage{footnote}}
\makesavenoteenv{longtable}
\usepackage{graphicx}
\makeatletter
\def\maxwidth{\ifdim\Gin@nat@width>\linewidth\linewidth\else\Gin@nat@width\fi}
\def\maxheight{\ifdim\Gin@nat@height>\textheight\textheight\else\Gin@nat@height\fi}
\makeatother
% Scale images if necessary, so that they will not overflow the page
% margins by default, and it is still possible to overwrite the defaults
% using explicit options in \includegraphics[width, height, ...]{}
\setkeys{Gin}{width=\maxwidth,height=\maxheight,keepaspectratio}
% Set default figure placement to htbp
\makeatletter
\def\fps@figure{htbp}
\makeatother

\usepackage{booktabs}
\usepackage{longtable}
\usepackage{array}
\usepackage{multirow}
\usepackage{wrapfig}
\usepackage{float}
\usepackage{colortbl}
\usepackage{pdflscape}
\usepackage{tabu}
\usepackage{threeparttable}
\usepackage{threeparttablex}
\usepackage[normalem]{ulem}
\usepackage{makecell}
\usepackage{xcolor}
\KOMAoption{captions}{tableheading}
\makeatletter
\makeatother
\makeatletter
\makeatother
\makeatletter
\@ifpackageloaded{caption}{}{\usepackage{caption}}
\AtBeginDocument{%
\ifdefined\contentsname
  \renewcommand*\contentsname{Table des matières}
\else
  \newcommand\contentsname{Table des matières}
\fi
\ifdefined\listfigurename
  \renewcommand*\listfigurename{Liste des Figures}
\else
  \newcommand\listfigurename{Liste des Figures}
\fi
\ifdefined\listtablename
  \renewcommand*\listtablename{Liste des Tables}
\else
  \newcommand\listtablename{Liste des Tables}
\fi
\ifdefined\figurename
  \renewcommand*\figurename{Figure}
\else
  \newcommand\figurename{Figure}
\fi
\ifdefined\tablename
  \renewcommand*\tablename{Table}
\else
  \newcommand\tablename{Table}
\fi
}
\@ifpackageloaded{float}{}{\usepackage{float}}
\floatstyle{ruled}
\@ifundefined{c@chapter}{\newfloat{codelisting}{h}{lop}}{\newfloat{codelisting}{h}{lop}[chapter]}
\floatname{codelisting}{Listing}
\newcommand*\listoflistings{\listof{codelisting}{Liste des Listings}}
\makeatother
\makeatletter
\@ifpackageloaded{caption}{}{\usepackage{caption}}
\@ifpackageloaded{subcaption}{}{\usepackage{subcaption}}
\makeatother
\makeatletter
\@ifpackageloaded{tcolorbox}{}{\usepackage[many]{tcolorbox}}
\makeatother
\makeatletter
\@ifundefined{shadecolor}{\definecolor{shadecolor}{rgb}{.97, .97, .97}}
\makeatother
\makeatletter
\makeatother
\ifLuaTeX
\usepackage[bidi=basic]{babel}
\else
\usepackage[bidi=default]{babel}
\fi
\babelprovide[main,import]{french}
% get rid of language-specific shorthands (see #6817):
\let\LanguageShortHands\languageshorthands
\def\languageshorthands#1{}
\ifLuaTeX
  \usepackage{selnolig}  % disable illegal ligatures
\fi
\usepackage[]{biblatex}
\addbibresource{stat.bib}
\IfFileExists{bookmark.sty}{\usepackage{bookmark}}{\usepackage{hyperref}}
\IfFileExists{xurl.sty}{\usepackage{xurl}}{} % add URL line breaks if available
\urlstyle{same} % disable monospaced font for URLs
\hypersetup{
  pdftitle={Grossesse\_Cx},
  pdfauthor={Philippe MICHEL},
  pdflang={fr-FR},
  colorlinks=true,
  linkcolor={blue},
  filecolor={Maroon},
  citecolor={Blue},
  urlcolor={Blue},
  pdfcreator={LaTeX via pandoc}}

\title{Grossesse\_Cx}
\usepackage{etoolbox}
\makeatletter
\providecommand{\subtitle}[1]{% add subtitle to \maketitle
  \apptocmd{\@title}{\par {\large #1 \par}}{}{}
}
\makeatother
\subtitle{Plan d'analyse statistique}
\author{Philippe MICHEL}
\date{13 octobre 2022}

\begin{document}
\maketitle
\ifdefined\Shaded\renewenvironment{Shaded}{\begin{tcolorbox}[interior hidden, enhanced, breakable, borderline west={3pt}{0pt}{shadecolor}, sharp corners, frame hidden, boxrule=0pt]}{\end{tcolorbox}}\fi

Ce document ne concerne que l'analyse statistique des données.

Le risque \(\alpha\) retenu sera de 0,05 \& la puissance de 0,8.

Des graphiques seront réalisés pour présenter les résultats importants.

\hypertarget{nombre-de-sujets-nuxe9cessaires.}{%
\section{Nombre de sujets
nécessaires.}\label{nombre-de-sujets-nuxe9cessaires.}}

Le critère principal est le choix d'un professionnel de santé par les
femmes pendant leur grossesse. Le choix possible comprend trois
possibilités :

\begin{itemize}
\tightlist
\item
  Médecin généraliste
\item
  Médecin gynécologue médical
\item
  Médecin gynécologue obstétricien
\item
  Sage-femme
\item
  Autre
\end{itemize}

Pour une première comparaison par un test de \(\chi^2\) un effectif de
400 cas parait raisonnable. Si on prévoit qu'un des groupe sera très
différents des autres 300 cas peuvent suffire. Mais ce premier résultat
ne permettra d'affirmer que l'existence probable d'une différence entre
les groupes (si le test est significatif) mais pas de dire qu'un groupe
est différent des quatre autres ou de classer les groupes. Pour cela il
faudra alors faire d'autres tests (\(\chi^2\) le plus souvent), répétés,
avec alors une perte de puissance.

Si le premier test ne renvoie pas une p.value significative, le résultat
est alors négatif \& l'analyse s'arrête là.

\hypertarget{description-de-la-population}{%
\subsection{Description de la
population}\label{description-de-la-population}}

\hypertarget{analyse-simple}{%
\subsubsection{Analyse simple}\label{analyse-simple}}

La description de la population concerne :

Plusieurs tableaux descriptifs simples seront réalisés. Des graphiques
de distribution pourront être réalisés pour les items les plus
importants.

Une recherche de corrélation entre les variable sera réalisée, celles-ci
devant être prises en compte pour l'analyse factorielle ou les
régressions.

\hypertarget{analyse-factorielle}{%
\subsubsection{Analyse factorielle}\label{analyse-factorielle}}

Si le nombre de cas recueillis le permet une analyse factorielle en MCA
(Analyse de correspondances multiples) sera réalisée. On essayera en
particulier de voir si les femmes ayant choisi d'être suivi par un MG
représente un groupe particulier.

Cette analyse ne pourra être qu'après transformation des variables
numériques en catégories \& imputation des données manquantes ce qui
n'est possible que si ces dernières ne sont pas trop nombreuses.

\hypertarget{objectif-principal}{%
\subsection{Objectif principal}\label{objectif-principal}}

L4analyse du critère principal (choix du professionnel par la femme)
sera réalisé par un test du \(\chi^2\) si les effectifs sont suffisants,
sinon par un test non paramétrique de Wilcoxon. Uniquement si ce premier
test global retourne une p.value significative des test complémentaires
seront effectués pour chercher si un choix est différent des autres, en
particulier suivi par son MG vs les autres professionnels.

Une analyse simple (test de \(\chi^2\) ou de Wilcoxon) sera réalisée
pour rechercher d'éventuels facteurs pronostics.

\hypertarget{analyse-par-ruxe9gression}{%
\subsubsection{Analyse par régression}\label{analyse-par-ruxe9gression}}

Une analyse multivariée par régression logistique sera réalisée en y
incorporant les toutes les variables ayant une p-value \textless{} 0,20
sur l'analyse monovariée.

Une analyse multivariée des courbes de survie (modèle de Cox) sera
réalisée sur les mêmes critères.

\hypertarget{objectifs-secondaires}{%
\subsection{Objectifs secondaires}\label{objectifs-secondaires}}

Les objectifs secondaires sont l'étude de facteurs influant cette
décision. Les comparaison seront faites par des tests de \(\chi^2\) pour
les onze items.

L'analyse du suivi par le médecin généraliste sera uniquement
descriptive. Si le nombre de femme ayant choisi le MG pour leur suivi de
grossesse est suffisant( \textgreater{} 30) une comparaison entre ce
groupe \& les autres participantes sera réalisé par des analyses par des
tests de \(\chi^2\).

\hypertarget{technique}{%
\subsection{Technique}\label{technique}}

L'analyse statistique sera réalisée avec le logiciel
\textbf{R}\autocite{rstat} \& divers packages en particulier
\texttt{tidyverse} \autocite{tidy}, \texttt{FactoMineR} \autocite{facto}
\& \texttt{epiDisplay} \autocite{epid}.


\printbibliography


\end{document}
